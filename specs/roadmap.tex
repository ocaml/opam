\documentclass[a4paper,11pt]{article}

\usepackage{url}
\usepackage{eurosym}
%\usepackage[french]{babel}
\usepackage[T1]{fontenc}
\usepackage{pdfswitch}
\usepackage{verbatim}
\usepackage{fullpage}

\newenvironment{remark}[1][Remark]{\begin{trivlist}
\item[\hskip \labelsep {\bfseries #1}]}{\end{trivlist}}

\title{
OPAM: a Package Management Systems for OCaml\\
Version 1.0.0 Roadmap\\ ~\ \\
THIS DOCUMENT IS A DRAFT\\
~\ \\}
\author{Thomas GAZAGNAIRE\\
\url{thomas.gazagnaire@ocamlpro.com}\\
}

\begin{document}

\maketitle

\vfill

\tableofcontents

\section*{Overview}

This document specifies the design of a package management system for
OCaml (OPAM). For the first version of OPAM, we have tried to consider
the simplest design choices, even if these choices restrict user
possibilities (but we hope not too much). Our goal is to propose a
system that we can build in a few months. Some of the design choices
might evolve to more complex tasks later, if needed. \\

A package management system has typically two kinds of users: {\em
  end-users} who install and use packages for their own projects; and
{\em packagers}, who create and upload packages. End-users want to
install on their machine a consistent collection of {\em packages} --
a package being a collection of OCaml libraries and/or programs.
Packagers want to take a collection of their own libraries and
programs and make them available to other developpers.

This document describes the fonctional requirements for both kinds of
users.

\subsection*{Conventions}

In this document, {\tt \$home}, {\tt \$opam}, {\tt \$lib}, {\tt
  \$bin}, {\tt \$build}, {\tt \$opamserver} and {\tt \$package} are
assumed to be set as follows:

\begin{itemize}

\item {\tt \$home} refers to the end-user home path, typically {\tt
  /home/thomas/} on linux, {\tt /Users/thomas/} on OSX {\tt
  C:\textbackslash Documents and Settings\textbackslash
  thomas\textbackslash} on Windows.

\item {\tt \$opam} refers to the filesystem subtree containing the
  client state. Default directory is {\tt \$home/.opam}.

\item {\tt \$lib} refers to where the end-user wants the libraries to
  be installed. Default directory is {\tt \$opam/OVERSION/lib} ({\tt
    OVERSION} is the OCaml compiler version).

\item {\tt \$bin} refers to where the end-user wants the binaries to
  be installed. Default directory is {\tt \$opam/OVERSION/bin} ({\tt
    OVERSION} is the OCaml compiler version).

\item {\tt \$build} refers to where packages are built before being
  installed. Default directory is {\tt \$opam/OVERSION/build} ({\tt
    OVERSION} is the OCaml compiler version).

\item {\tt \$doc} refers to where package documentation is
  installed. Default directory is {\tt \$opam/OVERSION/doc/} ({\tt
    OVERSION} is the OCaml compiler version).

\item {\tt \$opamserver} refers to the filesystem subtree containing
  the server state. Default directory is {\tt \$home/.opam-server}.

\item {\tt \$package} refers to a path in the packager filesystem, where
  lives the collection of libraries and programs he wants to package.

\end{itemize}

Variable are written in capital letters: for instance {\tt NAME}, {\tt
  VERSION}, {\tt OVERSION}, $\ldots$.

\section{Milestone 1: Foundations}

The first milestone of OPAM focuses on providing a limited set of
features, dedicated to package management of OCaml packages.  We limit
OPAM to support the installation of one version per packages only;
moreover, this first version of OPAM supports only one compiler
version.

\subsection{Client state}
\label{client}

The client state is stored on the filesystem, under {\tt \$opam}:

\begin{itemize}

\item {\tt \$opam/config} is the main configuration file. It defines
  the OPAM version, the repository addresses and the current compiler
  version. The file format is described in \S\ref{config}.

\item {\tt \$opam/index/NAME.VERSION.opam} are OPAM specification
  files for all available versions of all available packages. The
  format of OPAM files is described in \S\ref{opam}.

\item {\tt \$opam/descr/NAME.VERSION.txt} are textual files,
  containing the description for each available packages. The first
  line of this file is the package synopsis.

\item {\tt \$opam/OVERSION/installed} is the list of installed
  packages with their version for a given compiler version. The format
  of installed packages file is described in \S\ref{installed}.

\item {\tt \$opam/OVERSION/config/NAME.config} are package configuration
  files, containing environment variables for each installed
  packages. The file format is described in \S\ref{pconfig}.

\item {\tt \$opam/OVERSION/install/NAME.install} are package installation
  files, containing the installed files for each installed
  packages. The file format is described in \S\ref{install}.

\item {\tt \$opam/archives/NAME.VERSION.tar.gz} are source archives
  for all available versions of all available packages.

\item {\tt \$build/NAME.VERSION/} are tempory folders used to
  decompress the corresponding archives, for all the previously and
  currently installed package versions.

\item {\tt \$bin/} contains the installed binaries.

\item {\tt \$lib/NAME/} contains the installed libraries for the
  package {\tt NAME}.

\end{itemize}

\subsection{File Syntax}
\label{syntax}

\begin{description}

\item[Base types] The base types are:

\begin{itemize}
\item {\tt STRING} a doubly-quoted OCaml string, for instance: {\tt "foo"}
\item {\tt SYMBOL} a symbol contains only non-letter and non-digit
  characters, for instance: {\tt <=}
\item {\tt IDENT} an ident starts by a letter and is followed by any
  number of letters, digit and symbols, for instance: {\tt foo-bar}
\end{itemize}

\item[Compound types] Values of base types can be composed together to
  build complex types:

\begin{itemize}
\item {\tt [ X ]} a space-separated list of values of type {\tt X}
\item {\tt ( X )} a space-separated optional list of values of type
  {\tt X}
\item \verb+{ X }+ a space-separated collection of values of types
      {\tt X} (whose order is thus not meaningful).
\end{itemize}

\item[Files] All structured OPAM files share the same syntax:

\begin{itemize}

\item A file is a space-separated list of {\tt item}s

\item An {\tt item} is either:
\begin{itemize}
\item {\tt IDENT = value}
\item \verb+IDENT STRING { item }+
\end{itemize}

\item a {\tt value} is either:
\begin{itemize}
\item {\tt STRING}
\item {\tt SYMBOL}
\item {\tt [ VALUE ]}
\item \verb+VALUE ( VALUE )+
\end{itemize}

\end{itemize}
\end{description}

\subsubsection{Configuration files}
\label{config}

{\tt \$opam/config} has the following format:

\begin{verbatim}
opam-version = 1.0
sources = [ STRING ]
ocaml-version = STRING
\end{verbatim}

The field {\tt sources} contains the list of OPAM repositories
(default is {\tt "opam.ocamlpro.com"}). Initially, the field {\tt
  ocaml-version} corresponds to the output of {\tt 'ocamlc
  -version'}. \\

There are two kinds of repository sources:

\begin{itemize}

\item READ-ONLY repositories with {\tt https://}, {\tt http://} and
  {\tt ftp://} as prefix. They are synchronized using {\tt rsync}.

\item READ-WRITE repositories: they have {\tt opam://} as prefix. They
  are synchronized using a custom protocol. The server should run {\tt
    opam-server} on port 9999 to accept client connections.

%% \item {\bf (optional)} READ-ONLY repositories with {\tt git://} as
%%   prefix or {\tt .git} as suffix. They are synchronized using {\tt
%%   git}.

\end{itemize}

\subsubsection{Installed packages}
\label{installed}

{\tt \$opam/OVERSION/installed} has the following format:

\begin{verbatim}
STRING = STRING
STRING = STRING
...
\end{verbatim}

Each line \verb+NAME = VERSION+ in this file means that the version
{\tt VERSION} of package {\tt NAME} has been compiled with OCaml
version {\tt OVERSION} and has been installed on the system in {\tt
  \$opam/OVERSION/lib/NAME}.

\subsubsection{OPAM files}
\label{opam}

{\tt \$opam/index/NAME.VERSION.opam} has the following format:

\begin{verbatim}
opam-version = 1.0

package NAME {
  version     = STRING
  description = STRING
  maintainer  = STRING
  depends     = VALUE
  conflicts   = VALUE
  libraries  = [ STRING ]
  syntax     = [ STRING ]
}
\end{verbatim}

The first line specifies the OPAM version. \\

The contents of {\tt version} is {\tt VERSION}. The contents of {\tt
  description} is the name of the file, among the package files,
containing the package textual description. The first line of this
file is interpreted as the package synopsis. {\tt maintainer} contains
the contact address of the package maintainer. \\

The {\tt depends} and {\tt conflicts} fields contain expressions over
package names, optionally parametrized by version constrains. An
expression is either:

\begin{itemize}
\item A package name: {\tt "foo"};
\item A package name with version constraints:
  \verb+"foo" (>= "1.2" & <= "3.4")+
\item A disjunction of expressions: \verb+E | F+
\item A conjunction of expressions: \verb+E & F+
\item An expression with parenthesis: \verb+( E )+
\end{itemize}

For instance \verb+ "foo" (<= "1.2") & ("bar" | "gna" (= "3.14"))+
is a valid formula whose semantic is: {\em a version of package
  {\tt "foo"} lesser or equal to $1.2$ and either any version of
  package {\tt "bar"} or the version $3.14$ of package {\tt "gna"}.}
\\

The {\tt libraries} and {\tt syntax} fields contain the libraries and
syntax extensions defined by the package.

\subsection{Configuration files}
\label{pconfig}

\verb+$opam/OVERSION/config/NAME.config+ has the following syntax:

\begin{verbatim}
library STRING {
  include  = [ STRING ]
  asmlink  = [ STRING ]
  bytelink = [ STRING ]
  requires = [ STRING ( STRING ) ]
  pp       = [ STRING ( STRING ) ]
}

syntax STRING {
  include  = [ STRING ]
  asmlink  = [ STRING ]
  bytelink = [ STRING ]
  requires = [ STRING ( STRING ) ]
  pp       = [ STRING ( STRING ) ]
}

IDENT = STRING
IDENT = [ STRING ]
...
\end{verbatim}

Each {\tt library} and {\tt syntax} block defines full compile-time
options to use when linking with this library (not including the
dependencies options, which will be built dynamically using the {\tt
  requires} and {\tt pp} fields).\\

\begin{itemize}

\item {\tt include} is the list of directory to open when compiling a
  project using the library (or the syntax extension). It should at
  least contain \verb+[ "-I" "/full/path/to/NAME" ]+.

\item {\tt asmlink} is either the list of libraries to use when
  linking a project in native code with the library. It should at
  least contain \verb+[ "-I" "/full/path/to/NAME" "NAME.cmxa" ]+

\item {\tt asmlink} is either the list of libraries to use when
  linking a project in byte code with the library. It should at least
  contain \verb+[ "-I" "/full/path/to/NAME" "NAME.cma" ]+

\item {\tt requires} is the list of libraries which needs to be linked
  with the current one. The syntax \verb+"foo" ("bar" "gna")+ means
  only libraries {\tt "bar"} and {\tt "gna"} in package "foo" will be
  considered. The syntax {\tt "foo"} means {\em all} libraries in
  package {\tt "foo"} will be considered.

\item {\tt pp} is the list of syntax extension to use when compiling a
  program using the library. The syntax is similar to {\tt
    requires}. Once expended, the list of arguments is used with the
  {\tt -pp} command-line option of the chosen compiler.

\end{itemize}

The remaining fields {\tt IDENT = STRING} or {\tt IDENT = [ STRING ]}
are used to defined global variables associated to this package,
and are used to substitute variables in template files (using the
syntax \verb+%{PACKAGE}:VAR%+, see \S\ref{opam-config}.

XXXX


\subsubsection{Install files}
\label{install}

\verb+$opam/OVERSION/install/NAME.install+ has the following format:

\begin{verbatim}
lib  = [ "name.cmi" "name.cmo" "name.cmx" "name.o" ]
bin  = [ "foo.byte" ("foo") ]
doc  = [ "doc" ]
misc = [
  [ "foo.el" "/usr/share/emacs/site-lib"         ]
]
\end{verbatim}

Files listed under {\tt lib} should be copied to {\tt
  \$lib/NAME/}. File listed under {\tt bin} should be copied to {\tt
  \$bin/}. Files listed under {\tt doc} should be copied to {\tt \$doc/NAME/}.
Files listed under {\tt misc} should be processed as
follows: for each line {\tt FILE DST}, the tool should ask the user if
he wants to install {\tt FILE} to the absolute path {\tt DST}.

\subsection{Server state}

The filesystem of OPAM repositories are mirrored on the client
filesystem under {\tt \$opamserver/HOSTNAME} for each remote {\tt
  HOSTNAME}. This filesystem contains:

\begin{itemize}

\item {\tt \$opamserver/HOSTNAME/index/NAME.VERSION.opam}, which are
  OPAM files for all available versions of all available packages. The
  format of specification files is described in \S\ref{opam}.

\item {\tt \$opamserver/HOSTNAME/archives/NAME.VERSION.tar.gz} are the
  source archives for all available versions of all available
  packages.

\end{itemize}

Depending on the kind of OPAM repository, the most adapted
synchronization tools will be run between the server and client
filesystems. Moreover, the server files are installed at the right
place in the client state using symbolic links when it is possible
(ie. not on windows ...).

\subsection{Server API}
\label{api}

In this section all function are defined for a given OPAM repository.

\subsubsection{Basic types}

\begin{verbatim}
    type repo    = string
    type name    = string
    type version = string
    type opam
    type archive
\end{verbatim}

Names and versions are strings. Archives and specification files are
either binary strings or filenames.

\subsubsection{Getting the list of packages}
\label{getList}

\begin{verbatim}
    val getList: repo -> (name * version) list
\end{verbatim}

{\tt getList repo} updates the given repository and returns the list of
available versions for all packages.

\subsubsection{Getting specification files}
\label{getOpam}

\begin{verbatim}
    val getOPAM: repo -> (name * version) -> opam
\end{verbatim}

{\tt getOPAM repo (name,version)} returns the corresponding OPAM file.

\subsubsection{Getting package archive}
\label{getArchive}

\begin{verbatim}
    val getArchive: repo -> (name * version) -> archive
\end{verbatim}

{\tt getArchive repo (name,version)} returns the corresponding package
archive.

\subsubsection{Uploading new archives}
\label{newArchive}

\begin{verbatim}
    val newArchive: repo -> (opam * archive) -> unit
\end{verbatim}

{\tt newArchive(opam,archive)} takes as input an OPAM file and the
corresponding package archive, and upload the server state. This
function works only for READ-WRITE repository. In case of a READ-ONLY
one, a suitable error message is returned to the user.

\subsubsection{Binary Protocol}

In case of READ-WRITE repositories, the server state can be queried
and modified by any OPAM clients, using the following binary protocol

\begin{itemize}

\item Communication between clients and servers always start by an
hand-shake to agree on the protocol version.

\item All the basic values (names, versions and binary data) are
  represented as OCaml strings.

\item More complex values are marshaled using a simple binary
  protocol: the first byte represents the message number, and then
  each message argument is stacked in the message with its size as
  prefix. The list of messages {\em from the client to server} is:

{\small
\begin{tabular}{|l|l|l|}
\hline
Client-to-Server Message & Arguments & Description \\
\hline
\hline
\verb+GetList+ & -- & Ask for the list of all OPAM files \\
\hline
\verb+GetOPAM+ & \verb+name   : string+ & Ask for the binary representation of \\
               & \verb+version: string+ & a given OPAM file \\
\hline
\verb+GetArchive+ & \verb+name   : string+ & Ask for the binary representation of \\
                  & \verb+version: string+ & a given archive file \\
\hline
\verb+NewArchive+ & \verb+name   : string+ & Create a new package on the
server. \\
                  & \verb+version: string+ & The client should provide
the OPAM file \\
                  & \verb+opam   : string+ & and the source archive. \\
                  & \verb+archive: string+ & \\
\hline
\verb+UpdateArchive+ &  \verb+name   : string+ & Update a new version
of a given \\
                  & \verb+version: string+ & package on the
server. The client \\
                  & \verb+opam   : string+ & should also provide a security key\\
                  & \verb+archive: string+ & \\
                  & \verb+key    : string+ & \\
\hline
\end{tabular}
}

\item Answers from the server are encoded in the same way (ie, a byte
  for the message number, followed by optional arguments prefixed by
  their size). List arguments are encoded by stacking first the
  lenght, and then all the elements of the list in sequential order.
  The list of messages {\em from servers to clients} is:

{\small
\begin{tabular}{|l|l|l|}
\hline
Server-to-Client Message & Arguments & Description \\
\hline
\hline
\verb+GetList+      & \verb+list   : (string*string) list+ & Return the list
of available \\
 & & package names and versions \\
\hline
\verb+GetOPAM+      & \verb+opam   : string+ & Return an OPAM file \\
\hline
\verb+GetArchivwe+  & \verb+archive: string+ & Return an archive file \\
\hline
\verb+NewArchive+   & \verb+key    : string+ & Return a security key \\
\hline
\verb+UpdateArchive+& --                     & The update went OK \\
\hline
\verb+Error+        & \verb+error  : string+ & An error occurred \\
\hline
\end{tabular}
}

Note that when an error is raised by an arbitrary function
 at server side, the client receives \verb|Error _|.

\subsection{Client commands}

\subsubsection{Creating a fresh client state}

When an end-user starts OPAM for the first time, he needs to
initialize {\tt \$opam/} in a consistent state. In order to do so, he
should run:

\begin{verbatim}
    $ opam init [HOSTNAME]*
\end{verbatim}

Where {\tt HOSTNAME} are OPAM repositories. If no OPAM repository is
specified, default is {\tt opam://opam.ocamlpro.com}.

This command will:

\begin{enumerate}

\item create the file {\tt \$opam/config} containing:

\begin{verbatim}
version: 1.0
sources: [HOSTNAME]+
ocaml-version: OVERSION
\end{verbatim}

where {\tt OVERSION} is obtained by calling {\tt 'ocamlc -version}
(ie. we assume the user have already installed the OCaml compiler).

\item create an empty {\tt \$opam/OVERSION/installed} file.

\item ask the server for all available packages using {\tt
  getList} (\S\ref{getList}) and get all the corresponding spec files
  using {\tt getOpam} (\S\ref{getOpam}).

\item dump all the spec files into {\tt
  \$opam/index/NAME.VERSION.opam}.

\item create empty directories {\tt \$opam/archives}; and create {\tt
  \$lib} and {\tt \$bin} if they do not exist.

\end{enumerate}

\subsubsection{Listing packages}

When an end-user wants to have information on all available packages,
he should run:

\begin{verbatim}
    $ opam list
\end{verbatim}

This command will parse {\tt \$opam/OVERSION/installed} to know the
installed packages, and {\tt \$opam/index/*.opam} to get all the
available packages. It will then build a summary of each packages. For
instance, if {\tt batteries} version {\tt 1.1.3} is installed, {\tt
  ounit} version {\tt 2.3+dev} is installed and {\tt camomille} is not
installed, then running the previous command should display:

\begin{verbatim}
    batteries   1.1.3  Batteries is a standard library replacement
    ounit     2.3+dev  Test framework
    camomille      --  Unicode support
\end{verbatim}


In case the end-user wants a more details view of a specific package,
he should run:

\begin{verbatim}
    $ opam info NAME
\end{verbatim}

This command will parse {\tt \$opam/OVERSION/installed} to get the
installed version of {\tt NAME} and will look for {\tt
  \$opam/index/NAME.*.opam} to get available versions of {\tt
  NAME}. It can then display:

\begin{verbatim}
    package:  NAME
    version:  VERSION                 # '--' if not installed
    versions: VERSION1, VERSION2, ...
    description:
      LINE1
      LINE2
      LINE3
\end{verbatim}

\subsubsection{Installing a package}
\label{install}

When an end-user wants to install a new package, he should run:

\begin{verbatim}
    $ opam install NAME
\end{verbatim}

This command will:

\begin{enumerate}

\item look into {\tt \$opam/index/NAME.*.opam} to find the latest
  version of the package.

\item compute the transitive closure of dependencies and conflicts of
  packages using the dependency solver (see \S\ref{deps}). If the
  dependency solver returns more than one answer, the tool will ask
  the user to pick one, otherwise it will proceed directly.

\item the dependency solver should have sorted the collections of
  packages in topological order. Them, for each of them do:

\begin{enumerate}

\item check whether the package archive is installed by looking for
  the line {\tt NAME VERSION} in {\tt \$opam/OVERSION/installed}. If
  not, then:

\begin{enumerate}

\item look into the archive cache to see whether it has already been
  downloaded. The cache location is: {\tt
    \$opam/archives/NAME.VERSION.tar.gz}.

\item if not, then download the archive and store it in the cache.

\item decompress the archive into {\tt \$build/}. By convention, we
  assume that this should create {\tt \$build/NAME.VERSION/}.

\item run {\tt \$build/NAME.VERSION/build.sh}. By convention, package
  archives should contains such a file.

\item process {\tt
  \$build/NAME.VERSION/NAME.install}\label{NAME.install}.  The file
  format is described in \S\ref{install}.

\end{enumerate}
\end{enumerate}
\end{enumerate}

\begin{remark}
This installation scheme is not always correct, as installing a new
package should uninstall all packages depending on that one. For
instance, let us consider 3 packages {\tt A}, {\tt B} and {\tt C};
{\tt B} and {\tt C} depend on {\tt A}; {\tt C} depends on {\tt B}.
{\tt A} and {\tt B} are installed, and the user request {\tt C} to be
installed. If the version of {\tt A} is not correct one but the
version of {\tt B} is, the tool should: install the latest version of
{\tt A}, recompile {\tt B}, compile {\tt C}. It is understood that,
with this first milestone, {\tt B} will not be recompiled. This issue
will be fixed in next milestones of OPAM.
\end{remark}

\subsubsection{Updating index files}

When an end-user wants to know what are the latest packages available,
he will write:

\begin{verbatim}
    $ opam update
\end{verbatim}

This command will ask the server the list of available packages using
{\tt getList} (see \S\ref{getList}); then ask for the missing OPAM
files using {\tt getOpam} (see \S\ref{getOpam}). Finally it will dump
the missing OPAM files into {\tt \$opam/index/NAME.VERSION.opam}.

\subsubsection{Upgrading installed packages}

When an end-user wants to upgrade the packages installed on his host,
he will write:

\begin{verbatim}
    $ opam upgrade
\end{verbatim}

This command will call the dependency solver (see \S\ref{deps}) to
find a consistent state where {\em most} of the installed packages are
upgraded to their latest version. It will install each non-installed
packages in topological order, similar to what it is done during the
install step, See \S\ref{install}.

\subsubsection{Getting package configuration}

The first version of OPAM contains the minimal information to be able
to use installed libraries. In order to do so, the end-user (or the
packager) should run:

\begin{verbatim}
    $ opam config [-list|-var NAME:VAR|-subst FILENAME]
\end{verbatim}

This command will return:

\begin{itemize}
\item the list of all variables defined in installed packages
\item the content of variable {\tt VAR} defined in installed package
  {\tt NAME}
\item the file {\tt FILENAME} where every occurrence of
  \verb+%{NAME:VAR%}+ in {\tt FILENAME.in} is replaced by its content
\end{itemize}

XXXX


\subsubsection{Uploading packages}
\label{upload}

When a packager wants to create a package, he should:

\begin{enumerate}

\item create {\tt \$package/NAME.VERSION.opam} containing in the format
  specified in \S\ref{opam}.

\item create {\tt \$package/NAME.install} containing the list of files
  to install. File format is described in \ref{NAME.install});
  filnames should be relative to {\tt \$package}.

\item create the script {\tt ./build.sh} which will be called by the
  end-user installer. This script should configure and build the
  package on the end-user host.

\item create an archive {\tt NAME.VERSION.tar.gz} of the sources he
  wants to distribute, including {\tt \$NAME.install}, {\tt
    build.sh} and optionaly {\tt \$NAME.opam}.

\item run the following command:

\begin{verbatim}
    $ opam-upload NAME
\end{verbatim}

This command looks into the current directory for a file named {\tt
  NAME.opam}, and it will parse it to get the version number. Then it
looks in the current directory for the archive {\tt
  NAME.VERSION.tar.gz}. It will then use the server API~\ref{api} to
upload the package on the server.

\end{enumerate}

\subsection{Dependency solver}
\label{deps}

Dependency solving is a hard problem and we do not plan to start from
scratch implementing a new SAT solver. Thus our plan to integrate (as
a library) the Debian depency solver for CUDF files, which is written
in OCaml.

\begin{itemize}
\item the dependency solver should run on the client;
\item the dependency solver should take as input a list of packages
  (with some optional version information) the user wants to install
  and it should return a consistent list of packages (with version
  numbers) to install;
\item version information should be translated from arbitrary strings
  (used in OPAM files, see \S\ref{opam}) to integers (used by
  CUDF). We assume that version numbers are always incremented.
\item part of the input can be cached in {\tt \$opam/index.cudf}
  if necessary.
\end{itemize}

\section{Milestone 2: Correctness of Installation}

This milestone focus on correctness of installation and upgrade.

\subsection{Upgrading \& installing are always correct}

When the user wants to upgrade, he gets a list of packages in
topological order to install. When a package version is different from
the installed package version, the package should be built and should
replace the previous one. Then, all the packages depending on this
package should be recursively reinstalled (even if they have correct
version numbers).

\subsection{Removing packages}

When the user wants to remove a package, he should write:

\begin{verbatim}
    $ opam-remove NAME
\end{verbatim}

This command will check whether the package {\tt NAME} is installed,
and if yes, it will display to the user the list packages that will be
uninstalled (ie. the transitive closure of all forward-dependencies).
If the user accepts the list, all the packages should be uninstalled,
and the client state should be let in a consistent state.

\section{Milestone 3: Link Information}

This milestone focuses on adding the right level of linking
information, in order to be able to use packages more easily.

\subsection{Getting package link options}

The user should be able to run:

\begin{verbatim}
    $ opam-config -bytelink NAME
    $ opam-config -asmlink NAME
\end{verbatim}

This command will return the list of link options to pass to {\tt
  ocamlc} when linking with libraries exported by {\tt NAME}.

In order to be able to do so, packagers should provide a file {\tt
  NAME.descr} which gives link information such as:

\begin{verbatim}
    library foo {
      requires: bar, gni
      link: -linkall
      asmlink: -cclib -lfoo
     }
\end{verbatim}

\subsection{Getting package recursive configuration}

The user should be able to run:

\begin{verbatim}
    $ opam-config -r -dir NAME
    $ opam-config -r -bytelink NAME
    $ opam-config -r -asmlink NAME
\end{verbatim}

This command will return the good options to use for package {\tt
  NAME} and all its dependencies, in a form suitable to be used by
OCaml compilers.

\section{Milestone 4: Server Authentication}

This version focuses on server authentication.

\subsection{RPC protocol}

The protocol should be specified (using either a binary format or a
JSON format).

\subsection{Server authentication}

The server should be able to ask for basic credential proofs. The
protocol can be sketched as follows:

\begin{itemize}

\item packagers store keys in {\tt \$opam/keys/NAME}. These keys are
  random strings of size 128.

\item the server stores key hashes in {\tt
  \$opamserver/hashes/NAME}.

\item when a packager wants to upload a fresh package, he still uses
  {\tt newArchive}. However, the return type of this function is
  changed in order to return a random key. OPAM clients then stores
  that key in {\tt \$opam/keys/NAME}.

\item when a packager wants to uplaod a new version of an existing
  package, he uses the function {\tt val updateArchive: (opam * string
    * string) -> bool}. {\tt updateArchive} takes as argument an OCaml
  value representing the OPAM file contents, the archive file as a
  binary string and the key as a string. The server then checks
  whether the hash of the key is equal to the one stored in {\tt
    \$opamserver/hashes/NAME}; if yes, it updates the
  package and return {\tt true}, if no if it returns {\tt false}.

\item packager email should be specified in {\tt NAME.opam}:

\end{itemize}

\section{Milestones 6: Pre-Processors Information}

This milestone focus on the support of pre-processors.

\subsection{Getting package preprocessor options}

The user should be able to run:

\begin{verbatim}
    $ opam-config -bytepp NAME
    $ opam-config -asmpp NAME
\end{verbatim}

This command will return the command line option to build the
preprocessor exported by package {\tt NAME}.

In order to do so, packagers should describe exported preprocessors in
the corresponding {\tt NAME}.descr:

\begin{verbatim}
syntax foo {
  requires: bar, gni         // list of syntax dependencies
  pp: -parser o -printer p   // common options to asmpp and bytepp
  bytepp: ...
}
\end{verbatim}

\section{Milestones 7: Support of Multiple Compiler Versions}

This milestone focus on the support of multiple compiler versions.

\subsection{Compiler Description Files}

For each compiler version {\tt OVERSION}, the client and server states
will be extended with the following files:

\begin{itemize}
\item {\tt \$opam/compilers/OVERSION.comp}
\item {\tt \$opamserver/compilers/OVERSION.comp}
\end{itemize}

Each {\tt .comp} file contains:

\begin{itemize}

\item the location where this version can be downloaded. It can be an
  archive available via {\tt http} or using CVS such as {\tt svn} or
  {\tt git}.

\item eventual options to pass to the configure script. {\tt
  --prefix=\$opam/OVERSION/} will be automatically added to these
  options.

\item options to pass to {\tt make}.

\item eventual patch address, available via {\tt http} or locally on
  the filesystem

\end{itemize}

For instance, {\tt 3.12.1+memprof.comp} (OCaml version $3.12.1$ with
the memory profiling patch) looks like:

\begin{verbatim}
src:       http://caml.inria.fr/pub/distrib/ocaml-3.12/ocaml-3.12.1.tar.gz
build:     world world.opt
patches:   http://bozman.cagdas.free.fr/documents/ocamlmemprof-3.12.0.patch
\end{verbatim}

And {\tt trunk-tk-byte.comp} (OCaml from SVN trunk, with no {\em tk}
support and only in bytecode) looks like:

\begin{verbatim}
src:       http://caml.inria.fr/pub/distrib/ocaml-3.12/ocaml-3.12.1.tar.gz
configure: -no-tk
build:     world
\end{verbatim}

\subsection{Milestone 8: Version Pinning}

\subsection{Milestones 9: Parallel Build}

\subsection{Milestone 10: Version Comparison Scheme}

\subsection{Milestone 11: Database of Installed Files}


\end{document}
